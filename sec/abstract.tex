\begin{abstract}
	In this paper, we present a solution to the DEBS 2016 Grand Challenge that leverages Apache Flink, an open source platform for distributed stream and batch processing. The challenge focuses on the real-time analysis of an evolving social-network graph, to (1) determine the posts that trigger the current most activity, and (2) identify large communities that are currently involved in a topic.  
	
	We design the system architecture focusing on the exploitation of parallelism and memory efficiency so to enable an effective processing of high volume data streams on a distributed infrastructure.
	
	Our solution to the first query relies on a distributed and fine-grain approach for updating the post scores and determining partial ranks, which are then merged into a single final rank. Furthermore, changes in the final rank are efficiently identified so to update the output only if needed.
	
	The second query efficiently represents in-memory the evolving social graph and uses a customized Bron-Kerbosch algorithm to identify the largest communities active on a topic. We leverage on an in-memory caching system to keep the largest connected components which have been previously identified by the algorithm, thus saving computational time.
	
	We run the experiments on a single node equipped with 4 cores and 8 GB of RAM. For a medium dataset size with 200k events, our system can process up to 2.5(?) kEvents/s with an average latency of 0.4(?) ms for the first query, and up to 2.8(?) kEvents/s with an average latency of 0.5(?) ms for the second query. 	
\end{abstract}