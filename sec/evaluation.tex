\section{Evaluation}
\label{sec:evaluation}

We evaluated the performance of our solution on a single node equipped with 4 cores and 8 GB of RAM. We considered a target dataset consisting of 55906573 events, distributed as follows: 15\% posts, 44\% comments, 39\% likes and 2\% friendships. 

\begin{table}[h!]
	\centering
	\caption{Datasets}
	\label{tab:datasets}
	\begin{tabular}{l|c|c|}
		\cline{2-3}
		& percentage & events  \\ \hline
		\multicolumn{1}{|c|}{tiny}		& 1\%	& 559567	\\ \hline
		\multicolumn{1}{|c|}{xsmall}	& 10\%	& 5595659	\\ \hline
		\multicolumn{1}{|c|}{small}		& 25\%	& 13976645	\\ \hline
		\multicolumn{1}{|c|}{medium}	& 50\%	& 27953288	\\ \hline
		\multicolumn{1}{|c|}{large}		& 75\%	& 41929932	\\ \hline
		\multicolumn{1}{|c|}{target}	& 100\%	& 55906573	\\ \hline
	\end{tabular}
\end{table}

During the experimentation phase, we considered the target dataset variants shown in Table \ref{tab:datasets}, which retain the original events distribution.